%===================================================================================
% JORNADA CIENTÍFICA ESTUDIANTIL - MATCOM,
%===================================================================================
% Esta plantilla ha sido diseñada para ser usada en los artículos de la
% Jornada Científica Estudiantil, MatCom.
%
% Por favor, siga las instrucciones de esta plantilla y rellene en las secciones
% correspondientes.
%
% NOTA: Necesitará el archivo 'jcematcom.sty' en la misma carpeta donde esté este
%       archivo para poder utilizar esta plantila.
%===================================================================================



%===================================================================================
% PREÁMBULO
%-----------------------------------------------------------------------------------
\documentclass[a4paper,10pt,twocolumn]{article}

%===================================================================================
% Paquetes
%-----------------------------------------------------------------------------------
\usepackage{amsmath}
\usepackage{amsfonts}
\usepackage{amssymb}
\usepackage{jcematcom}
\usepackage[utf8]{inputenc}
\usepackage{listings}
\usepackage[pdftex]{hyperref}
%\usepackage{stix}
%-----------------------------------------------------------------------------------
% Configuración
%-----------------------------------------------------------------------------------
\hypersetup{colorlinks,%
	    citecolor=black,%
	    filecolor=black,%
	    linkcolor=black,%
	    urlcolor=blue}

%===================================================================================



%===================================================================================
% Presentacion
%-----------------------------------------------------------------------------------
% Título
%-----------------------------------------------------------------------------------
\title{Segmentaci\'on de \'Ulceras de Pie Diab\'etico}

%-----------------------------------------------------------------------------------
% Autores
%-----------------------------------------------------------------------------------
\author{Jes\'us Aldair Alfonso P\'erez \\
\name  \email \href{mailto:aldairalfonsoperez@gmail.com}{aldairalfonsoperez@gmail.com}
    \\ \addr C-312  \AND
Mauro Jos\'e Bolado Vizoso \\
\name   \email \href{mailto:maurovizoso@gmail.com}{maurovizoso@gmail.com}
    \\ \addr C-311}

%-----------------------------------------------------------------------------------
% Tutores
%-----------------------------------------------------------------------------------
\tutors{\\
Alejandro Mesejo, \emph{Universidad de La Habana} \\}


%-----------------------------------------------------------------------------------
% Headings
%-----------------------------------------------------------------------------------
\jcematcomheading{\the\year}{1-\pageref{end}}{Modelos Matem\'aticos Aplicados}

%-----------------------------------------------------------------------------------
\ShortHeadings{}{Autores}
%===================================================================================



%===================================================================================
% DOCUMENTO
%-----------------------------------------------------------------------------------
\begin{document}

%-----------------------------------------------------------------------------------
% NO BORRAR ESTA LINEA!
%-----------------------------------------------------------------------------------
\twocolumn[
%-----------------------------------------------------------------------------------

\maketitle

%===================================================================================
% Resumen y Abstract
%-----------------------------------------------------------------------------------
\selectlanguage{spanish} % Para producir el documento en Español

%-----------------------------------------------------------------------------------
% Resumen en Español
%-----------------------------------------------------------------------------------
\begin{abstract}

	La segmentaci\'on de im\'agenes es el proceso de particionar una imagen digital 
	en m\'ultiples segmentos de im\'agenes, tambien conocidas como regiones en im\'agenes 
	o objetos en im\'agenes. Este \'articulo hace \'enfasis en espacios de colores y en una 
	aplicaci\'on de corte en grafos. Se presenta dos algoritmo para segmentar \'ulceras de pie 
	diab\'etico, se realiza una comparaci\'on de los resultados obtenidos entre los algoritmo 
	propuestos. Se concluye que el algoritmo basado en corte en grafos es m\'as eficiente que 
	el basado en colores.

\end{abstract}

%-----------------------------------------------------------------------------------
% English Abstract
%-----------------------------------------------------------------------------------
\vspace{0.5cm}

\begin{enabstract}
	
	Image segmentation is the process of partitioning a digital image into multiple segments 
	also known as image regions or objects. This article focuses on color spaces and on a 
	graph cut application. Two algorithms are presented to segment diabetic foot ulcers, a 
	comparison of the results obtained between the proposed algorithms is made. It is concluded 
	that the algorithm based on graph cut is more efficient than the one based on colors.

\end{enabstract}

%-----------------------------------------------------------------------------------
% Palabras clave
%-----------------------------------------------------------------------------------
\begin{keywords}
DFU: Diabetic Foot Ulcers

\end{keywords}

%-----------------------------------------------------------------------------------
% Temas
%-----------------------------------------------------------------------------------
\begin{topics}
Segmentaci\'on de \'Ulceras de Pie Diab\'etico
\end{topics}


%-----------------------------------------------------------------------------------
% NO BORRAR ESTAS LINEAS!
%-----------------------------------------------------------------------------------
\vspace{0.8cm}
]
%-----------------------------------------------------------------------------------


%===================================================================================

%===================================================================================
% Introducción
%-----------------------------------------------------------------------------------
\section{Introducción}\label{sec:intro}
%---------------------------------------------------------------

% An introduction in spanish about image segmentation

La segmentaci\'on


Aqu\'{i} se introduce la tem\'{a}tica del trabajo, motivaci\'{o}n, al final se explica cu\'{a}l ser\'{a} la estructura, es decir c\'{o}mo han subdividido su trabajo, importante lo que est\'{e} tomado de la literatura deben referenciarlo en el lugar que corresponda dentro del cuerpo del trabajo y listar la referencias al final. Todas las referencias listadas deben estar referenciadas en alguna parte del trabajo.
Para elegir un formato pde escritura para las referencias, ponen el t\'{i}tulo del trabajo en google scholar y abren Cite donde est\'{a}n las comillas y elegin un formato y luego donde dice Bibtex, y solo tienen que copiar y pegar.

%===================================================================================



%===================================================================================
% Desarrollo
%-----------------------------------------------------------------------------------
\section{Desarrollo}\label{sec:dev}
%-----------------------------------------------------------------------------------
A partir de aqu\'{i} comienza el desarrollo del trabajo, que pueden dividir en secciones y subsecciones y que si necesitan las pueden etiquetar.

Generalmente uno dedica las primeras secciones a presentar el problema que va a resolver, a explicar elementos y resultados te\'{o}ricos que ser\'{a}n usados en el trabajo. Luego una secci\'{o}n para explicar la propuesta de soluci\'{o}n.
En este contexto aparece siempre una secci\'{o}n de experimentaci\'{o}n, donde se explican los experimentos que se realizan y se exponen los resultados y en una subsecci\'{o}n se discuten los resultados, No es lo mismo exponer los resultados que discutirlos, generalmente la discusi\'{o}n lleva comparaciones con el estado del arte, pero como estos trabajos con peque\~{n}os se har\'{a} seg\'{u}n  vean con sus tutores.
Las conclusiones no pueden faltar, y siempre es interesante oir recomendaciones
 
 

%-----------------------------------------------------------------------------------
	\subsection{}\label{}


%-----------------------------------------------------------------------------------
\subsection{}




%-----------------------------------------------------------------------------------




%-----------------------------------------------------------------------------------


%-----------------------------------------------------------------------------------



%----------------------------------------------------------------------------

%-----------------------------------------------------------------------------------

 
 



%-----------------------------------------------------------------------------------






%-----------------------------------------------------------------------------------

\section{Experimentaci\'{o}n y resultados}
\subsection{Discusi\'{o}n}

%===================================================================================
% Conclusiones
%-----------------------------------------------------------------------------------
\section{Conclusiones}\label{sec:conc}
 
 



 

%===================================================================================





%===================================================================================
% Bibliografía
%-----------------------------------------------------------------------------------
\begin{thebibliography}{99}
%-----------------------------------------------------------------------------------
	\bibitem{}

%-----------------------------------------------------------------------------------
\end{thebibliography}

%-----------------------------------------------------------------------------------

\label{end}

\end{document}

%===================================================================================
